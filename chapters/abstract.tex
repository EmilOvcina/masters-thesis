\section*{\centering Abstract}
Microservices architectures are the prominent paradigm for developing distributed systems.
Large companies like Netflix and Uber utilize this architecture to handle large amounts of traffic.
Jolie is a programming language used to develop microservice applications.
Everything in Jolie is contained within services, and the developer defines the \emph{application programming interface} (API) for each service directly in the code.
Microservice API patterns are architectural patterns which can be used to enhance a microservice system.
These patterns can be used to create specific behaviour on the architectural level.

Jolie services alone cannot determine the behaviour of the architecture, so the developer of the system needs to have a good understanding of the topology of the system and how the different microservices communicate.
Several visualization tools exist which allow the developer to model and diagram the architecture to convey key information to others.
This requires the developer to have a good understanding of the system being developed.

A tool has been developed to aid Jolie developers understand the architecture they are developing. The tool visualizes the topology of the system and allows the developer to 
get an overview of key information about each component. The tool's functionality is further extended to allow the developer to refactor the code directly in the tool's user interface, as well as, "apply" a microservice API pattern to a selected set of services.

\section*{\centering Resumé}
Microservice-arkitekturer er det fremtrædende paradigme inden for udvikling af distribuerede systemer.
Store virksomheder som Netflix og Uber udnytter denne arkitektur til at håndtere store mængder af trafik.
Jolie er et programmeringssprog, der bruges til at udvikle microservice-applikationer.
Alt i Jolie er indeholdt i tjenester, og udvikleren definerer en \emph{application programming interface} (API) for hver tjeneste i koden.
Microservice-API-mønstre er arkitekturmønstre, der bruges til at forbedre et microservicesystem.
Disse mønstre kan bruges til a lave specifik adfærd på et arkitekturniveau.

Jolie-tjenester kan ikke bestemme adfærden af arkitekturen alene, så udvikleren af systemet skal have en god forståelse for systemets topologi og hvordan de forskellige microtjenester kommunikerer.
Der findes flere visualiseringsværktøjer, der giver udvikleren mulighed for at modellere og lave diagrammer for at formidle arkitekturen og vigtig information til andre.
Dette kræver at udvikleren har en god forståelse af det system, der bliver udviklet.

Et værktøj er udviklet til at hjælpe Jolie-udviklere med at skabe overblik over den arkitektur som de udvikler.
Værktøjet visualiserer systemets topologi og giver udvikleren et overblik over vigtig information om hvert komponent.
Værktøjets funktionalitet er yderligere udvidet, så udvikleren har mulighed for at refaktorere koden direkte i værktøjets \emph{user interface} samt "anvendelse" af et microservice-API-mønstre på en udvalgt mængde af tjenester.