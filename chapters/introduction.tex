% Text:              √
% Citations          X
% Revision:          X
% Final Revision:    X
\chapter{Introduction}
% history
Microservice architectures are the prominent paradigm for developing distributed applications which utilize scalability, maintainability and tight development and deployment cycles.
Microservice architectures have become popular and are utilized by companies like Uber and Netflix
to handle the amount of traffic from users each day.\footnote{The story of Netflix and Microservices - \url{https://www.geeksforgeeks.org/the-story-of-netflix-and-microservices/}}\footnote{Uber microservice architecture - \url{https://scaleyourapp.com/an-insight-into-how-uber-scaled-from-a-monolith-to-a-microservice-architecture/}}

% the issue
The challenges with microservice architectures are mainly the complexity. It can be very difficult for an inexperienced developer to start making an application using microservices and still have an overview of how the system works and what APIs the different services require and exposes.
Managing a microservice architecture which in many cases can be complex and dynamic requires the developers and architects to utilize dedicated tools, technologies, and methodologies.
Jolie\cite{jolie} is a service-oriented language which is one of the solutions to help developers create and manage microservice architectures, while microservice API patterns\cite*{PatternsForAPIDesign:2022} are a set of patterns the developer can utilize to create well-functioning architectures.
However, Jolie works at the service level and microservice API patterns work at the architectural level. It can be difficult to bridge this gap.

% dissatisfaction
Visualization is one solution to bridge the gap and aid the developer in understanding the architecture.
Most current tools for visualization require the developer to have an understanding of the architecture to visualize it.
These tools are the diagramming tools which aim to assist the developers in conveying the architecture to others but not so much to the developer.

\section{Scope \& Aim}
The thesis aims to provide developers with a tool for visualizing and refactoring microservice architectures written in Jolie. The tool seeks to aid the developer with understanding the architecture while they are developing their applications and also help them maintain complex projects.
The visualization capabilities entail the possibility for the user to see how the different services communicate and display key information about each component.
The refactoring capabilities entail the possibility for the user to easily edit the key information of the components, as well as, apply microservice patterns to the current architecture or a set of services.

The scope of the tool will be to have the visualization capabilities implemented.
To make sure that time constraints are not violated, only some of the refactoring capabilities will be implemented but will be extendable if more time is put into the development.
Lastly, a showcasing of how microservice patterns can be applied to the architecture using the tool will also be implemented, again with the possibility to extend the number of patterns that can be applied.

% \section{Structure of the Thesis}
% Firstly, the necessary background information will be highlighted, then an extensive and illustrative example will be used to introduce the reader to the tool and its capabilities.
% After the reader has been familiarized with the tool, the implementation and design decisions will be discussed and explained. The process of developing the tools will also be discussed in order to highlight some of the
% tested methods in order to end up with the current version of the tool.
